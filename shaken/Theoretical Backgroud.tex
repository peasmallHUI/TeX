\documentclass{article}
\usepackage{amsmath}
\usepackage{ctex}

\title{Shaken Optical Lattice Theoretical Backgroud}
\author{Xiao-Hui, HU}
\date{\today}

\begin{document}
	\maketitle
	电磁场对原子作用力可以表示为,这个力由原子所在处的光与原子相互作用能的梯度产生:
	\begin{equation}
		\boldsymbol{F} = -\nabla U(\boldsymbol{r},t) \label{eq:1}
	\end{equation}
	$U$是电磁场与原子相互作用能。它与原子所在位置$\boldsymbol{r}$和时间$t$有关,一般可以写成
	\begin{equation}
		U = - \boldsymbol{p} \cdot \boldsymbol{E}(\boldsymbol{r},t) \label{eq:2}
	\end{equation}
	%$\boldsymbol{p}$为感生偶极矩。

	这个力本质上就是电磁场的洛伦兹力(包含电场和磁场力)。它是由不可分割的电场力$\boldsymbol{f}_e$和磁场力$\boldsymbol{f}_m$组成:
	\begin{equation}
		\boldsymbol{f}_e = \rho \boldsymbol{E}(\boldsymbol{r},t) \label{eq:3}
	\end{equation}
	\begin{equation}
		\boldsymbol{f}_m = \rho \frac{d\boldsymbol{r}}{dt} \times \boldsymbol{B}(\boldsymbol{r},t) \label{eq:4}
	\end{equation}
	由于电磁场是周期振动的,频率很高,原子实际受到的力是对瞬间力在周期$T$取平均值:
	\begin{equation}
		\bar{\boldsymbol{F}} = \frac{1}{T} \int_{0}^{T} \boldsymbol{F}(t)dt \label{eq:5}
	\end{equation}
	我们有$\boldsymbol{F_e} = e \boldsymbol{E}(\boldsymbol{r},t)$。因为在原子中$\boldsymbol{r}$与$\boldsymbol{R}$相距很小,可对$\boldsymbol{E}$在$\boldsymbol{R}$附近展开,
	所以得到$\boldsymbol{E}$在$O_xyz$坐标系中各个分量
	\begin{equation}
		E_i(\boldsymbol{r},t) = E_i(\boldsymbol{R},t) + \sum_j (r_j - R_j) \frac{\partial E_i(\boldsymbol{R},t)}{\partial r_j} + \cdots  \label{eq:6}
	\end{equation}
	\centerline{$(i,j = x,y,z)$}

	我们可只取一次项,而忽略高次项,因为对电场
	\begin{equation*}
		E(\boldsymbol{r},t) = Re\{Ee^{i(\omega t-\boldsymbol{k} \cdot \boldsymbol{r})}\} 
	\end{equation*}
	的展开式按$kr=2\pi r /\lambda$的幂进行,$r/\lambda$的最大值为原子半径与电磁场波长之比,$a/\lambda$是约为$10^{-3}$数量级的小数。这样,我们有
	\begin{equation}
		F_{ei}(\boldsymbol{R},t) = eE_i(\boldsymbol{R},t) + \sum_j e(r_j - R_j) \frac{\partial E_i(\boldsymbol{R},t)}{\partial r_j} \label{eq:7}
	\end{equation}
	对振动周期进行平均后,等号右边的第一项消去为零,第二项中的$e(r_j - R_j)$(即电偶极矩$p_j$)是一个表示原子内部状态的量。
	将偏微分算符简化,引入
	\begin{equation*}
		\partial_j E_i = {\left[\frac{\partial E_i(\boldsymbol{r},t)}{\partial_{r_j}}\right]}_{\boldsymbol{r}=\boldsymbol{R}},
	\end{equation*}
	得到
	\begin{equation}
		\overline{F_{ei}} = \overline{\sum_j p_j(\partial_j E_i)} \label{eq:8}
	\end{equation}
	同样,从式\ref{eq:4}可得
	\begin{equation}
		\boldsymbol{F}_m = \frac{d\boldsymbol{p}}{dt} \times \boldsymbol{B}(\boldsymbol{R},t) \label{eq:9}
	\end{equation}
	这里我们将$\boldsymbol{r}$换成$\boldsymbol{R}$。这种近似的根据是磁场力比电荷作用的电场力要小$v/c$数量级($v$是原子运动速度,$c$是光速),高阶项就更可不考虑了。
	因为有
	\begin{align*}
		\frac{d}{dt}(\boldsymbol{p} \times \boldsymbol{B}) &= \frac{d\boldsymbol{p}}{dt} \times \boldsymbol{B} + \boldsymbol{p} \times \frac{d\boldsymbol{B}}{dt} \\
		&= \frac{d\boldsymbol{p}}{dt} \times \boldsymbol{B} + \boldsymbol{p} \times \left[ \frac{\partial \boldsymbol{B}}{\partial t} + \left(\frac{d\boldsymbol{R}}{dt} \cdot \nabla_{\boldsymbol{R}}\right)\boldsymbol{B}(\boldsymbol{R},t)\right]
	\end{align*}
	由Maxwell方程中$\frac{\partial \boldsymbol{B}}{\partial t} = -\nabla \times \boldsymbol{E}$;而又因为$\frac{d\boldsymbol{R}}{dt} \ll c$,后一项可忽略。这样,式\ref{eq:9}化为
	\begin{equation}
		\boldsymbol{F}_m = \frac{d}{dt}(\boldsymbol{p} \times \boldsymbol{B}) + \boldsymbol{p} \times (\nabla \times \boldsymbol{E}) \label{eq:10}
	\end{equation}
	又根据梯度算符$\nabla$的矢量运算公式
	\begin{equation*}
		\nabla(\boldsymbol{A} \cdot \boldsymbol{B}) = (\boldsymbol{A} \cdot \nabla)\boldsymbol{B} + (\boldsymbol{B} \cdot \nabla)\boldsymbol{A} + \boldsymbol{A} \times (\nabla \times \boldsymbol{B}) + \boldsymbol{B} \times (\nabla \times \boldsymbol{A})
	\end{equation*}
	微分只对场起作用,对偶极矩的微分可忽略。将式\ref{eq:10}展开,简化偏微分算符,得到
	\begin{equation}
		F_{mi} = \frac{d}{dt}(\boldsymbol{p} \times \boldsymbol{B})_i + \sum_j p_j(\partial_i E_j) - \sum_j p_j(\partial_j E_i) \label{eq:11}
	\end{equation}
	按振动周期平均后第一项消去,于是有
	\begin{equation}
		\overline{F_{mi}} = \overline{\sum_j p_j(\partial_i E_j)} - \overline{\sum_j p_j(\partial_j E_i)} \label{eq:12}
	\end{equation}
	式中等号右边第一项与式\ref{eq:8}相同。联合两式,可得洛伦兹力的$i$分量
	\begin{equation}
		\overline{F_{mi}} = \overline{\sum_j p_j(\partial_i E_j)} \label{eq:13}
	\end{equation}
	对坐标的偏微分只对场起作用,所以上式的完整矢量表示可写成
	\begin{equation}
		\overline{\boldsymbol{F}} = \overline{\nabla(\boldsymbol{p} \cdot \boldsymbol{E})} \label{eq:14}
	\end{equation}
	这就是式\ref{eq:1}和\ref{eq:2}的合成结果,不过$\boldsymbol{E}(\boldsymbol{r},t)$换成了$\boldsymbol{E}(\boldsymbol{R},t)$,而梯度算符$\nabla$只对电场起作用。
	式\ref{eq:14}表面上不含磁场,但实际上已经包含磁场作用。
	在量子力学中用上式可写成
	\begin{equation}
		\boldsymbol{F} = \langle p \nabla E \rangle = \langle p \rangle \langle \nabla_{\boldsymbol{R}}E(\boldsymbol{R},t) \rangle \label{eq:15}
	\end{equation}
	如果另电场的形式为
	\begin{align}
		E(\boldsymbol{R},t) &= E_0(\boldsymbol{R})\cos[\omega t + \phi(\boldsymbol{R})] \notag \\
		&= E_0^+(\boldsymbol{R})e^{-i\omega t} + E_0^-(\boldsymbol{R})e^{i\omega t} \label{eq:16}
	\end{align}
	其中
	\begin{equation}
		E_0^+ = \frac{1}{2}E_0(\boldsymbol{R})e^{-i\phi(\boldsymbol{R})},  E_0^- = \frac{1}{2}E_0(\boldsymbol{R})e^{i\phi (\boldsymbol{R})} \label{eq:17}
	\end{equation}
	由于
	\begin{equation}
		\langle p \rangle = tr(\rho p) = p(\rho_{12} + \rho_{21}) \label{eq:18}
	\end{equation}
	其中,$\rho$为密度矩阵,并应用$p_{12} = p_{21} = p,p_{11} = p_{22} = 0$。再利用旋转波近似去掉以光频高速变化的部分。为此令
	\begin{equation}
		\rho_{21}(t) = \sigma_{21}(t)e^{-i\omega t}, \rho_{12}(t) = \rho_{21}^*(t) = \sigma_{12}(t)e^{i \omega t} \label{eq:19}
	\end{equation}
	为了统一,也把$\rho_{11}$、$\rho_{22}$改成
	\begin{equation}
		\rho_{11} = \sigma_{11}, \rho_{22} = \sigma_{22} \label{eq:20}
	\end{equation}
	将式\ref{eq:19}和式\ref{eq:20}代入式\ref{eq:15},忽略以$2\omega t$呈指数变化的高频项,得到
	\begin{equation}
		\boldsymbol{F}(\boldsymbol{R},t) = p(\sigma_{12} \nabla E_0^+ + \sigma_{21} \nabla E_0^-) \label{eq:21}
	\end{equation}
	由于这里涉及相位问题,我们将对计算稍作变动。引入变换
	\begin{equation}
		\sigma_{21}'= \sigma_{21} e^{i\phi(\boldsymbol{R},t)}, \sigma_{12}'= \sigma_{12} e^{-i\phi(\boldsymbol{R},t)} \label{eq:22}
	\end{equation}
	将上式和式\ref{eq:17}带入式\ref{eq:21}并引入拉比频率$\Omega = pE_0(\boldsymbol{R})/\hbar$,可得
	\begin{equation}
		\boldsymbol{F}(\boldsymbol{R},t) = \hbar (u\nabla \Omega + v\Omega \nabla \phi)/2 \label{eq:23}
	\end{equation}
	其中
	\begin{equation}
		u = \sigma_{12}' + \sigma_{21}', v = -i(\sigma_{12}' - \sigma_{21}')
	\end{equation}
	再引入
	\begin{equation}
		\omega = \sigma_{22} - \sigma_{11}
	\end{equation}
	得到三个参量的时间变化率公式
	\begin{equation}
		\frac{du}{dt} = (\delta + \frac{d\phi}{dt})v - \frac{\Gamma}{2}u 
	\end{equation}
	\begin{equation}
		\frac{dv}{dt} = -(\delta + \frac{d\phi}{dt})u + \Omega\omega - \frac{\Gamma}{2}v 
	\end{equation}
	\begin{equation}
		\frac{d\omega}{dt} = - \Omega v - \Gamma(\omega + 1)
	\end{equation}
	这是光学布洛赫方程的另一种形式,其中$u,v$代表以外电场频率振动的感生偶极矩的两个相互垂直的分量,而$\omega$则表示上、下能级的原子布居差。
	我们只求方程的稳态解,并认为开始时原子是静止的,这样令方程组等号左边均为零,且有
	\begin{equation}
		\frac{d\phi}{dt} = \nabla \phi(\boldsymbol{R}) \cdot \frac{d\boldsymbol{r}}{dt} = \nabla \phi \cdot \boldsymbol{v} = 0
	\end{equation}
	可以解得
	\begin{equation}
		u = \frac{-\delta \Omega}{\delta^2 + \Gamma^2/4 + \Omega^2/2}
	\end{equation}
	\begin{equation}
		v = \frac{-\Gamma \Omega/2}{\delta^2 + \Gamma^2/4 + \Omega^2/2}
	\end{equation}
	\begin{equation}
		\omega = \frac{\delta^2 + \Gamma^2/4}{\delta^2 + \Gamma^2/4 + \Omega^2/2}
	\end{equation}
	将上述解带入到式\ref{eq:23},可以得到电磁场中原子所受到的力
	\begin{equation}
		\boldsymbol{F} = - \frac{\hbar \Omega^2}{2} \frac{(\omega - \omega_a) \nabla \Omega/\Omega + \Gamma \nabla \phi/2}{(\omega - \omega_a)^2 + (\Gamma/2)^2 + \Omega^2/2}
	\end{equation}
	由上式可见,这个力由两部分组成,即$\boldsymbol{F} = \boldsymbol{F_1} + \boldsymbol{F_2}$,我们把第一部分称为散射力,第二部分称为偶极力($\boldsymbol{F_1}$代表单位时间内原子得到的光子总动量;$\boldsymbol{F_2}$方向决定于电磁场的频率失谐,偶极矩和不均匀场是产生这个力的充要条件)。
	当原子以速度$v_0$运动时,上面有关原子受力公式中的原子中心位置$\boldsymbol{R}$就成为时间$t$的函数:
	\begin{equation}
		\boldsymbol{R} = \boldsymbol{R}_0 + \boldsymbol{v}_0t
	\end{equation}
	现在设激光场是沿$z$轴偏振且沿$x$轴传播的驻波场:
	\begin{equation}
		E_x(x,t) = 2E_0\cos(kx)\cos(\omega t)
	\end{equation}
	这个光场的振幅是随坐标变化的,但相位处处相同,$\nabla \phi = 0$。因此原子受力只有偶极力部分:
	\begin{equation}
		F_2 \propto \frac{\nabla \Omega}{\Omega} = \frac{\nabla E(x,t)}{E(x,t)} \propto -tan(kx)
	\end{equation}
	在场强很小时,有$\Omega \ll \Gamma$。这时,力的表达式可以简化,得到:
	\begin{equation}
		F = \frac{\hbar k \Omega^2 \delta}{\delta^2 + \Gamma^2/4} \left\{ sin(2kx) + kv_0 \frac{\Gamma}{\delta^2 + \Gamma^2/4}[1-\cos(2kx)] \right\} \label{eq:37}
	\end{equation}
	在一定的光强分布下对偶极力求积分,就可以得到阱势的形式
	\begin{equation}
		V_{dip}(r) = \int_0^{\infty} F_2 dr
	\end{equation}
	为了得到较强的势阱,要求驻波光的光强很大。设原子的速度很小,从强光修正的偶极力式\ref{eq:37}可以得到势阱的公式:
	\begin{equation}
		V = \frac{v_0)}{2} \cos(2kx)
	\end{equation}
	式中势阱深度$V_0$与激光光强成正比。

	\clearpage
	两束相向的激光可以有两种方法实现,第一,将一束激光分裂成两束;第二,通过镜子反射激光。前者很有可能在两束激光之间产生频率差$\Delta \nu t$,如:
	\begin{align}
		E_1(x,t) &= E_0 \cos(kx + \omega t + \Delta \nu t) \notag \\
		E_2(x,t) &= E_0 \cos(kx - \omega t) 
	\end{align}
	则,合成光场为:
	\begin{equation}
		E(x,t) = 2E_0 \cos(kx + \frac{\Delta \nu}{2}t) \cos(\omega t + \frac{\Delta \nu}{2}t)
	\end{equation}
	如果分裂光束差距$\Delta \nu(t)$,则整个晶格会以速度$v(t) = d_L\Delta \nu(t) $移动或振动。其中$d_L = \pi / k$。因此在实验坐标系下势阱的形式为:
	\begin{equation}
		V_{lab}(x,t) = \frac{V_0}{2} \cos(2k[x-X_0(t)])
	\end{equation}
	其中,
	\begin{equation}
		X_0(t) = d_L \int_0^t d \tau \Delta \nu(\tau)
	\end{equation}
	如果通过镜面反射也可以改变$X_0(t)$的形式来达到晶格振动的目的。
	那么原子在振动光晶格中的哈密顿量为:
	\begin{equation}
		H_{lab}(t) = \frac{p^2}{2M} + V_{lab}(x,t)
	\end{equation}
	利用幺正变换可将哈密顿形式转化为下列形式:
	\begin{equation}
		H = H_0 + H_1
	\end{equation}
	其中
	\begin{equation}
		H_0 = \frac{p^2}{2M} + \frac{V_0}{2}\cos(2kx)
	\end{equation}

	\begin{equation}
		H_1 = -F(t)x 
	\end{equation}
	作为外力,
	\begin{align}
		F(t) &= -M \ddot x \notag \\
			 &= -M \frac{\partial v}{\partial t} \notag \\
			 &= -M d_L \frac{\Delta \nu}{\Delta t}
	\end{align}
	现假设$\Delta \nu = \Delta \nu_{max} \sin(\omega t)$,则我们可以得到外力情况为:
	\begin{align}
		F(t) &= M d_L \frac{d}{dt}[\Delta \nu_{max} \sin(\omega t)] \notag \\
			 &= M d_L \omega \Delta \nu_{max} \cos(\omega t)
	\end{align}
\end{document}