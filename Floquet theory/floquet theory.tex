\documentclass{article}
\usepackage{amsmath}
\usepackage{ctex}
\usepackage{mathdots}


\title{Floquet theory and quasi-energy}
\author{Xiao-Hui, HU}
\date{\today}

\begin{document}
    \maketitle
    
    Floquet理论主要处理的是时间周期下的哈密顿量的问题。
    该系统的薛定谔方程为:
    \begin{equation}
        i \partial_{t} | \psi(t) \rangle=H(t) | \psi(t) \rangle, \quad H(t+T)=H(t)
    \end{equation}

    我们这里引入Floquet算符和Floquet态,其中floquet算符是在一个周期内的时间演化算符。
    \begin{equation}
        F[t]=U(t+T, t), \quad F[t] | \psi(t) \rangle=| \psi(t+T) \rangle
    \end{equation}

    而Floquet态为Floquet算符的本征态:
    \begin{equation}
        F[t] | \Psi_{\alpha}(t) \rangle=\lambda_{\alpha} | \Psi_{\alpha}(t) \rangle
    \end{equation}

    因为演化算符是幺正的,因此$\lambda_{\alpha}=e^{i \theta_{\alpha}}$作为相位因子。固其应满足:
    \begin{equation}
        | \Psi_{\alpha}(t+T) \rangle=e^{i \theta_{\alpha}} | \Psi_{\alpha}(t) \rangle
    \end{equation}

    我们可对Floquet态取以下形式:
    \begin{equation}
        | \Psi_{\alpha}(t) \rangle=e^{-i \varepsilon_{a} t} | u_{\alpha}(t) \rangle, \quad | u_{\alpha}(t+T) \rangle=| u_{\alpha}(t) \rangle
    \end{equation}

    其中$| u_{\alpha}(t) \rangle$被称作Floquet模,它与哈密顿量的时间周期一致。这里可以看出Floquet态与Bloch态的相似性。
    它们一个为时间周期,一个为空间周期。可以验证得到该形式满足时间周期关系:
    \begin{equation}
        F[t] | \Psi_{\alpha}(t) \rangle=| \Psi_{\alpha}(t+T) \rangle=e^{-i \varepsilon_{a} T} | \Psi_{\alpha}(t) \rangle
    \end{equation}

    这里$ \varepsilon_{a}$与定态中能量类似,称为准能量。将上述形式带入薛定谔方程,可以得到:
    \begin{equation}
        \left\{H(t)-i \partial_{t}\right\} | u_{\alpha}(t) \rangle=\varepsilon_{\alpha} | u_{\alpha}(t) \rangle
    \end{equation}
    为了能够利用定态问题的讨论方式,我们这里决定将希尔伯特空间扩展。首先扩展能量和Floquet模:
    \begin{equation}
        \overline{\varepsilon}_{\alpha m}=\varepsilon_{\alpha}+m \omega, \quad | \overline{u}_{\alpha m}(t) \rangle=e^{i m \omega t} | u_{\alpha}(t) \rangle
    \end{equation}

    可以得出扩展后的Floquet态与原来类似:
    \begin{equation}
        | \Psi_{\alpha}(t) \rangle=e^{-i \varepsilon_{a} t} | u_{\alpha}(t) \rangle=e^{-i \overline{\varepsilon}_{\alpha m} t} | \overline{u}_{\alpha m}(t) \rangle
    \end{equation}

    不管m取任何值,$\left\{\varepsilon_{\alpha m}, u_{\alpha m}\right\}$都对应同一Floquet态。
    同时,进一步引入:
    \begin{equation}
        Q(t)=H(t)-i \partial_{t}
    \end{equation}

    这样可以将薛定谔方程化简为:
    \begin{equation}
        Q(t) | \overline{u}_{\alpha m}(t) \rangle=\overline{\varepsilon}_{\alpha m} | \overline{u}_{\alpha m}(t) \rangle
    \end{equation}

    并且在该空间重新定义新的内积:
    \begin{equation}
        \left\langle\left\langle\overline{u}_{\alpha m} | \overline{v}_{\beta n}\right\rangle\right\rangle=\frac{1}{T} \int_{0}^{T} d t\left\langle\overline{u}_{\alpha m}(t) | \overline{v}_{\beta n}(t)\right\rangle
    \end{equation}
    这保证了不同的m的Floquet模是正交的。

    对于扩展哈密顿量的矩阵形式,先选定基矢:
    \begin{equation}
        | \lambda m \rangle \rangle=| \lambda \rangle e^{i m \omega t}
    \end{equation}

    我们可以得到扩展哈密顿量的矩阵元:
    \begin{equation}
    \begin{aligned}\langle\langle\lambda m|Q| \xi n\rangle\rangle &=\frac{1}{T} \int_{0}^{T} d t e^{-i m \omega t}\left\langle\lambda\left|\left\{H(t)-i \partial_{t}\right\}\right| \xi\right\rangle e^{i n \omega t} \\ &=\left\langle\lambda\left|H_{m-n}\right| \xi\right\rangle+ m \omega \delta_{\lambda \xi} \delta_{m n} \end{aligned}
    \end{equation}

    其基本结构可以表示为如下形式:
    \begin{equation}
        \left\{
            \begin{matrix}
            \ddots  &               &           &               & \iddots    \\ 
                    & H_0 + \omega  &   H_1     & H_2           &           \\
                    & H_{-1}        &   H_0     & H_1           &           \\
                    & H_{-2}        &   H_{-1}  & H_0 - \omega  &           \\
            \iddots  &               &           &               & \ddots    
            \end{matrix}
        \right\}
    \end{equation}
    
    这样可以通过对角化该矩阵来得到准能谱。
    
    现假设某一系统的哈密顿如下形式:
    \begin{equation}
        \hat{H}(t)=\frac{\hat{p}^{2}}{2 M}+V_{0} \cos ^{2}\left(k_{L} x\right)-F_{0} \sin (\omega t)
    \end{equation}

    通过欧拉变换$\sin (\omega t)=1 /(2 i)\left(e^{i \omega t}-e^{-i \omega t}\right)$,
    我们可以重写哈密顿量:
    \begin{equation}
        \hat{H}(t)=\sum_{m=-1}^{1} \hat{H}_{m} e^{i m \omega t}
    \end{equation}

    这样我们可以得到$\hat{H}_{0} = \frac{\hat{p}^{2}}{2 M}+V_{L}(x)$,
    且$\hat{H}_{ \pm 1} = \pm F_{0} /(2 i)$。这样可以利用定态形式下的解法来解决时间周期下的问题。



\end{document}